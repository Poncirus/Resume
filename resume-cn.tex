% !TEX program = xelatex
\documentclass{resume}
\usepackage{lastpage}
\usepackage{fancyhdr}
\usepackage{linespacing_fix}
\usepackage[fallback]{xeCJK}

\pagestyle{fancy}
\fancyhead{}
\fancyfoot{}
\fancyfoot[R]{2021.10.25}

\begin{document}

\renewcommand\headrulewidth{0pt}
\name{廖瀚文}

% fill space
\vspace{6pt}

\basicInfo{
    \email{work@liaohanwen.com}\textperiodcentered\
    \phone{(+86) 136-3833-2881}\textperiodcentered\
    \github[Poncirus]{https://github.com/Poncirus}\textperiodcentered\
    \homepage[liaohanwen.com]{https://liaohanwen.com}\textperiodcentered\
    \linkedin[Hanwen Liao]{https://www.linkedin.com/in/hanwen-liao-4043b8196/}
}

% fill space
\vspace{6pt}

\section{教育经历}
\datedsubsection{\textbf{纽约大学}, 美国}{2019.09 -- 2021.01}
专业:Computer Engineering(计算机工程)
\datedsubsection{\textbf{北京邮电大学}, 北京}{2015.09 -- 2019.06}
专业:通信工程

% fill space
\vspace{6pt}

\section{工作经历}
\datedsubsection{\textbf{猿辅导}, 北京}{2021.03 -- 2021.09}
软件开发工程师 - 基础研发部,直播中台,数据平台组
\begin{itemize}[parsep=0.25ex]
    \item 使用Doris,Flink,Spark,Iceberg等工具进行大数据分析,完成多个需求的Java Web开发,通过报表等形式呈现直播中台相关数据指标,为直播中台提供数据支持
    \item 负责维护和迭代数据查询与可视化工具Redash,主要包括新数据源的支持,SQL编写方式和图表配置方式的扩展,报警功能的优化,内网登录的接入
    \item 负责以Netty为框架的客户端打点采集服务,完成日常服务优化,进行服务打压和容量估计
    \item 优化打点重复和丢失的问题,提出优化方案并领导执行,问题用户比例从千分之几降低到万分之一
    \item 设计客户端抽样上报打点的方案和协议,领导实现相关功能
    \item 截取Sentry Kafka数据,对客户端崩溃信息进行分类,并结合灰度开关等指标进行展示
    \item 完成客户端百万QPS打点日志从阿里云日志服务到Elasticsearch的迁移
\end{itemize}
\datedsubsection{\textbf{华为}, 深圳}{2020.08 -- 2020.12}
软件开发工程师(实习) - 2012实验室,中央软件院,庞加莱实验室
\begin{itemize}[parsep=0.25ex]
    \item 分析C/C++语言程序的函数调用关系,了解LLVM编译原理
    \item 开发基于LLVM的变量追踪脚本
\end{itemize}
\datedsubsection{\textbf{东信北邮信息技术有限公司}, 北京}{2018.06 -- 2019.08}
软件开发工程师(实习) - ISMP部门
\begin{itemize}[parsep=0.25ex]
    \item 使用C++维护ISMP(综合服务管理平台)
    \item 开发ISMP接口模块,编解码HTTP消息和JSON消息
    \item 修改Lua编译器,增加注解的语法,通过修改词法分析器禁用指定变量名
    \item 重构消息分发方式,完成负载均衡,提高可用性
    \item 使用Golang开发ISMP与微服务平台的接口,传递JSON消息
\end{itemize}

% fill space
\vspace{6pt}

\section{个人项目}
\datedsubsection{\textbf{智能健康项目}}{2018.10 -- 2019.04}
无线通信教研中心,北京邮电大学
\begin{itemize}[parsep=0.25ex]
    \item 与研究生小组一起完成智能床垫的开发
    \item 开发和维护Django服务端程序,从设备获取数据并使用MySQL数据库存储,响应客户端的请求
    \item 开发配套安卓APP,从服务端获取数据并通过Chart.js使用图表展示数据
\end{itemize}
\datedsubsection{\textbf{IoT节点设备嵌入式开发}}{2018.06 -- 2018.10}
百科融创科技发展有限公司
\begin{itemize}[parsep=0.25ex]
    \item 在STM开发板上完成IoT节点程序开发
    \item 开发配套安卓APP,从开发板获取数据并使用图表展示
\end{itemize}

% fill space
\vspace{6pt}

\section{技能}
\begin{itemize}[parsep=0.25ex]
    \item
          \textbf{编程语言}:
          Java, C/C++, Golang, Python, SQL, Sell Script
    \item
          \textbf{其他技术:}:
          Git, Gerrit, Linux
\end{itemize}
\end{document}
